\section{Business Plan}
\subsection{Introduction}
Many home and business owners have received electric bills and have sometimes wondered why it was so high. They generally wonder if they just simply used more energy that month or if one habit or appliance was to blame. However, they rarely see these questions answered. The PICA system will answer these questions without needing to go through the hassle of calling the power company and having a time consuming conversation. This feasibility study was conducted in order to gauge if the design team could develop this system in the time table given.  

\subsection{System Feasibility}
The PICA system separates into three subsystems that will communicate with each other to form one system. The team of four electrical engineers brings many different talents, experiences, and interests to contribute to the project. Each subsystem divides to cater to each team member’s area of interest and talent. By separating the system into different subsystems, it increases the feasibility of the system as a whole. This allows for each team member to do research, testing, and assembly separately. It also allows each team member to gauge time for parts to be delivery. The final step will be making sure each subsystem can communicate with each other. However, if the design team addresses these issues early, there should be fewer problems in the future. 

\subsection{Market Feasibility}
Not only will the PICA system monitor total power consumption, but it will be monitoring power on a circuit-by-circuit basis. This functionality distinguishes the PICA system from anything else on the market now, which only allow for plug-by-plug consumption or whole home consumption. Circuit-by-circuit monitoring allows the consumer to identify their energy hogs without having to test each device individually. This functionality also makes the market feasibility much greater. Another feature that increases the market feasibility is the fact that the PICA system will also have the capability of notifying the consumer when peak pricing hours of the day are, which is when the power company charges more per kilowatt-hour. 

\subsection{Legal Feasibility}
The PICA system will comply with all necessary codes and compliances, as well as keep the customer’s information private. Many different implementations accomplish these legal requirements. The operating system will only give administrator privileges to the power-company and customers, which the application of the system specifies. The user interface will have a web interface for system administration, but the information will only be accessible to an authenticated administrator. The PICA system also needs to meet certain codes in order to be safe enough for the customer to use, which will also protect from unexpected lawsuits. Underwriters Laboratories (UL) is an independent product safety certification organization, which offers safety certifications to products\cite{UL_Web}.  The system will be UL certifiable. The system will also restrict electromagnetic (EM) radiation to comply with FCC Title 47 Part 15. It will also be compliant with ANSI C12.19 and ANSI C12.21 standards. The PICA system will also safely isolate high-voltage areas.
 
\subsection{Schedule Feasibility}
The timeline of the project is eight months. During these eight months, the team members will develop a prototype to test and make necessary corrections based on testing results. The feasibility of this schedule is within the range of time needed to complete this project. Team members holding other team members accountable for their respective tasks maintain this schedule. The system divides into sub-systems for each team member to work on also contributes to a feasible eight-month schedule. The team has milestones in a Gantt chart to help keep an accurate account of the timeline. Additionally, there is a project manager specifically responsible to keep tasks updated and keep team members on track.

\subsection{Resource Feasibility}
There are many resources available to the team that can contribute to the project. There are many different facilities to have space to work. People and company resources are great for the design team as well, because the experience level is much higher. Consumer's Energy has offered to be available for questions, should any arise. Mark Mitchmerhuizen is head engineer at Johnson Controls Incorporated and has volunteered to be a mentor to the design team. Calvin College has an electronic shop with free parts that are available for students to use. For the parts that are not available in the electronics shop, Calvin College gives the design team a three hundred dollar budget to use at their disposal. This could be a potential problem to feasibility, because due to the size of the project, the cost might exceed three hundred dollars. To account for this the team members are acquiring some part donations from companies willing to help.

