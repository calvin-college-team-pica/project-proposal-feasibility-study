\documentclass[11pt,letterpaper,titlepage,toc]{article}
% Change "article" to "report" to get rid of page number on title page
\usepackage{amsmath,amsfonts,amsthm,amssymb}
\usepackage[margin=1in]{geometry}
\usepackage{setspace}
\usepackage[T1]{fontenc}                                              %Provides T1 Font encoding for European characters
\usepackage{Tabbing}
\usepackage{array}                                                         % Allows for making new column types
\usepackage{dcolumn}					        % Provides decimal alignment column macros
\usepackage{fancyhdr}                                                  % Produce a "fancy" header using \rhead,\chead,\lhead
\usepackage{lastpage}
\usepackage{extramarks}
\usepackage{chngpage}
\usepackage{chngcntr} 						%figure and equation numbering by section
\usepackage{soul}
\usepackage{bookmark}                                                 % Prevents having to re-run due to labels
\usepackage[usenames,dvipsnames]{color}
\usepackage{graphicx,float,wrapfig}
\usepackage{tocloft}							% Gives us the . . . in the TOC always and forever
\usepackage{ifthen}
\usepackage{listings}						% Allows for Listings
\usepackage{courier}                                                       % Provides the courier Font
\usepackage{subfig}
\usepackage{float}
\usepackage[siunitx,arrowmos]{circuitikz}
\usepackage{outlines}
\usepackage{pslatex}                                                    % This should switch the font to Times New Roman
%\usepackage[scaled=.92]{helvet}
\usepackage[printonlyused,withpage]{acronym}      % Creates acronym table of acronyms and definitions used
									        % in this document. \begin{acronym}[TDMA] ... \end{acronym}								   
\usepackage[final]{pdfpages}
%\usepackage{natbib}                                                   % Turn off for IEEE references
\usepackage{tikz-timing}
\usepackage{pdflscape}
\usepackage{multicol}
\usepackage{wrapfig}
\usepackage{longtable}						% Provides the Longtable environment
\usepackage{colortbl}						% Provides \rowcolor{}
\usepackage[parfill]{parskip} 					% remove the dumb indents on all paragraphs
\usepackage[toc,page]{appendix}				% Provides more control over the Appendix
\usepackage{hyperref}						% Hyperlink output PDF Files
\usepackage[all]{hypcap}						% Fix the dumb hyperlink bug!!!

%\input{kvmacros}



%%%%%%%%%%%%%%%%%%%%%%%%%%%%%%%%%%%%%%%%%%%%%%%%
% TOC dots
\renewcommand{\cftsecleader}{\cftdotfill{\cftdotsep}}
%%%%%%%%%%%%%%%%%%%%%%%%%%%%%%%%%%%%%%%%%%%%%%%%
% Setup the header and footer
\pagestyle{fancy}                                                       %
\lhead{Team PICA}                                                 %
\rhead{PPFS -- Draft}  %
\chead{}                                                     %
\lfoot{\lastxmark}                                                      %
\cfoot{}                                                                %
\rfoot{Page\ \thepage\ of\ \pageref{LastPage}}                          %
\renewcommand\headrulewidth{0.4pt}                                      %
\renewcommand\footrulewidth{0.4pt}                                      %

\fancypagestyle{plain}{
\lhead{Team PICA}                                                 %
\rhead{PPFS -- Draft}  %
\chead{}                                                     %
\lfoot{\lastxmark}                                                      %
\cfoot{}                                                                %
\rfoot{}                          %
\renewcommand\headrulewidth{0.4pt}                                      %
\renewcommand\footrulewidth{0.4pt}}

\definecolor{MyDarkGreen}{rgb}{0.0,0.4,0.0}
%%%%%%%%%%%%%%%%%%%%%%%%%%%%%%%%%%%%%%%%%%%%%%%%%
% Creating special column types -- mostly math types
%%%%%%%%%%%%%%%%%%%%%%%%%%%
% Start by defining some macros
\newcolumntype{d}[1]{D{.}{\cdot}{#1}}
\newcolumntype{.}{D{.}{.}{-1}}
\newcolumntype{,}{D{,}{,}{2}}
\newcolumntype{C}{>{$}c<{$}}
\newcolumntype{R}{>{$}r<{$}}
%%%%%%%%%%%%%%%%%%%%%%%%%%%%%%%%%%%%%%%%%%%%%%%%%
% For faster processing, load Matlab syntax for listings
\lstloadlanguages{MATLAB, [x86masm]Assembler, C++, VHDL}%
\lstset{frame=single,                                                                              % Single frame around code
        basicstyle=\tiny\ttfamily,             					   % Use small true type font
        keywordstyle=[1]\color{blue}\ttfamily,        				   % MATLAB functions bold and blue
        keywordstyle=[2]\color{purple},         				            % MATLAB function arguments purple
        keywordstyle=[3]\color{blue}\underbar,  			            % User functions underlined and blue
        identifierstyle=,                       						   % Nothing special about identifiers
                                                							   % Comments small dark green courier
        commentstyle=\usefont{T1}{pcr}{m}{sl}\color{MyDarkGreen}\scriptsize,
        stringstyle=\color{purple},             					   % Strings are purple
        showstringspaces=false,                 					   % Don't put marks in string spaces
        tabsize=5,                              						   % 5 spaces per tab
        %
        %%% Put standard MATLAB functions not included in the default
        %%% language here
        morekeywords={ra,stw, addi, ldw, bge, beq, br},
        %
        %%% Put MATLAB function parameters here
        morekeywords=[2]{on, off, interp},
        %
        %%% Put user defined functions here
        morekeywords=[3]{FindESS, homework_example},
        %
        morecomment=[l][\color{blue}]{...},     % Line continuation (...) like blue comment
        numbers=left,                           % Line numbers on left
        firstnumber=1,                          % Line numbers start with line 1
        numberstyle=\tiny\color{black},          % Line numbers are blue
        stepnumber=5,                            % Line numbers go in steps of 5
        breaklines=true,
        breakatwhitespace=false
        }  
 


% This is used to trace down (pin point) problems
% in latexing a document:
%\tracingall
\hfuzz2pt % I don't want to know about overfull hbox if < 2pt
%%%%%%%%%%%%%%%%%%%%%%%%%%%%%%%%%%%%%%%%%%%%%%%%%%%%%%%%%%%%%
% Some tools
\newcommand{\enterProblemHeader}[1]{\nobreak\extramarks{#1}{#1 continued on next page\ldots}\nobreak%
                                    \nobreak\extramarks{#1 (continued)}{#1 continued on next page\ldots}\nobreak}%
\newcommand{\exitProblemHeader}[1]{\nobreak\extramarks{#1 (continued)}{#1 continued on next page\ldots}\nobreak%
                                   \nobreak\extramarks{#1}{}\nobreak}%

\newlength{\labelLength}
\newcommand{\labelAnswer}[2]
  {\settowidth{\labelLength}{#1}%
   \addtolength{\labelLength}{0.25in}%
   \changetext{}{-\labelLength}{}{}{}%
   \noindent\fbox{\begin{minipage}[c]{\columnwidth}#2\end{minipage}}%
   \marginpar{\fbox{#1}}%

   % We put the blank space above in order to make sure this
   % \marginpar gets correctly placed.
   \changetext{}{+\labelLength}{}{}{}}%

\newcommand{\homeworkProblemName}{}%
\newcommand{\homeworkShortProblemName}{}
\newcounter{homeworkProblemCounter}%
\newenvironment{homeworkProblem}[2]
  {\stepcounter{homeworkProblemCounter}%
   \renewcommand{\homeworkProblemName}{#1}
   \renewcommand{\homeworkShortProblemName}{#2}
   \section*{\homeworkProblemName\ -- \homeworkShortProblemName}%
   \addcontentsline{toc}{section}{\homeworkShortProblemName}%
   \enterProblemHeader{\homeworkProblemName}}%
  {\exitProblemHeader{\homeworkProblemName}}%

\newcommand{\problemAnswer}[1]
  {\noindent\fbox{\begin{minipage}[c]{\columnwidth}#1\end{minipage}}}%

\newcommand{\problemLAnswer}[1]
  {\labelAnswer{\homeworkProblemName}{#1}}

\newcommand{\homeworkSectionName}{}%
\newlength{\homeworkSectionLabelLength}{}%
\newenvironment{homeworkSection}[1]%
  {% We put this space here to make sure we're not connected to the above.
   % Otherwise the changetext can do funny things to the other margin

   \renewcommand{\homeworkSectionName}{#1}%
   \settowidth{\homeworkSectionLabelLength}{\homeworkSectionName}%
   \addtolength{\homeworkSectionLabelLength}{0.25in}%
   \changetext{}{-\homeworkSectionLabelLength}{}{}{}%
   \subsection*{\homeworkSectionName}%
   \addcontentsline{toc}{subsection}{\homeworkSectionName}%
   \enterProblemHeader{\homeworkProblemName\ [\homeworkSectionName]}}%
  {\enterProblemHeader{\homeworkProblemName}%

   % We put the blank space above in order to make sure this margin
   % change doesn't happen too soon (otherwise \sectionAnswer's can
   % get ugly about their \marginpar placement.
   \changetext{}{+\homeworkSectionLabelLength}{}{}{}}%

\newcommand{\sectionAnswer}[1]
  {% We put this space here to make sure we're disconnected from the previous
   % passage

   \noindent\fbox{\begin{minipage}[c]{\columnwidth}#1\end{minipage}}%
   \enterProblemHeader{\homeworkProblemName}\exitProblemHeader{\homeworkProblemName}%
   \marginpar{\fbox{\homeworkSectionName}}%

   % We put the blank space above in order to make sure this
   % \marginpar gets correctly placed.
}%
   
\newcounter{subsubsubsection}[subsubsection]
\def\subsubsubsectionmark#1{}
\def\thesubsubsubsection {\thesubsubsection
     .\arabic{subsubsubsection}}
\def\subsubsubsection{\@startsection
     {subsubsubsection}{4}{\z@} {-3.25ex plus -1
     ex minus -.2ex}{1.5ex plus .2ex}{\normalsize\sf}}
% mj02r: original:
%\def\l@subsubsubsection{\@dottedtocline{4}
%     {4.8em}{4.2em}}
% mj02r: for VCE reports:
%\def\l@subsubsubsection{\@dottedtocline{4}
%     {7em}{3.8em}}
% mj02r, 29/12/2004: for thesis:
\def\l@subsubsubsection{\@dottedtocline{4}
     {11.1em}{4.6em}}

\makeatletter
\renewcommand{\paragraph}{\@startsection{paragraph}{4}{0ex}%
   {-3.25ex plus -1ex minus -0.2ex}%
   {1.5ex plus 0.2ex}%
   {\normalfont\normalsize\bfseries}}
\makeatother
   
% Includes a MATLAB script.
% The first parameter is the label, which also is the name of the script
%   without the .m.
% The second parameter is the optional caption.
\newcommand{\matlabscript}[2]
  {\begin{itemize}\item[]\lstinputlisting[language=MATLAB,caption=#2,label=#1]{#1.m}\end{itemize}}  
  
\newcommand{\ccode}[2]
  {\begin{itemize}\item[]\lstinputlisting[language={C++},caption=#2,label=#1]{#1.c}\end{itemize}}  
    
\newcommand{\assemblycode}[2]
  {\begin{itemize}\item[]\lstinputlisting[language={[x86masm]Assembler},caption=#2,label=#1]{#1.s}\end{itemize}}
  
 
\newcommand{\vhdlcode}[2]
  {\begin{itemize}\item[]\lstinputlisting[language={[AMS]VHDL},caption=#2,label=#1]{#1.vhd}\end{itemize}} 
  
\newcommand{\degree}{$^{\circ}$}
\renewcommand{\textbeta}{$\beta\ $}
\def\tm{\leavevmode\hbox{$\rm {}^{TM}$}}

%%%%%%%%%%%%%%%%%%%%%%%%%%%%%%%%%%%%%%%%%%%%%%%%%%%%%%%%%%%%%
%%%%%%%%%%%%%%%%%%%%%%%%%%%%%%%%%%%%%%%%%%%%%%%%
%This sets up the counters for the figures and equations to be numbered inside their sections
%\counterwithin{figure}{subsection}
%\counterwithin{equation}{section}
%\counterwithin{table}{subsection}
%\setcounter{secnumdepth}{5}
%\counterwithin{lst}{homeworkProblemCounter}
%%%%%%%%%%%%%%%%%%%%%%%%%%%%%%%%%%%%%%%%%%%%%%%%

%%%%%%%%%%%%%%%%%%%%%%%%%%%%%%%%%%%%%%%%%%%%%%%%%%%%%%%%%%%%%
% Make title
%\title{\vspace{2in}\textmd{\textbf{\hmwkAuthorName}}\\\normalsize\vspace{0.1in}\hmwkClass:\ \hmwkTitle \vspace{0.1in}\\Due\ on\ \hmwkDueDate\\\vspace{0.1in}\large{\textit{\hmwkClassInstructor\ \hmwkClassTime}}\vspace{3in}}
%\date{}
%\author{}
%\title{\vspace{2in}\textmd{\textbf{\hmwkClass:\ \hmwkTitle}}\\\normalsize\vspace{0.1in}\small{Due\ on\ \hmwkDueDate}\\\vspace{0.1in}\large{\textit{\hmwkClassInstructor\ \hmwkClassTime}}\vspace{3in}}
%\date{}
%\author{\textbf{\hmwkAuthorName}}

% IEEE Title
\title{Team PICA Project Proposal and Feasibility Study}
\date{\today}
\author{Amy Ball, Nate Jen, Avery Sterk, Kendrick Wiersma}
%%%%%%%%%%%%%%%%%%%%%%%%%%%%%%%%%%%%%%%%%%%%%%%%%%%%%%%%%%%%%

\raggedright
\setcounter{tocdepth}{4}
\addtocontents{toc}{\protect\thispagestyle{fancy}}

\begin{document}

\maketitle
\newpage
%\setcounter{page}{0}
\tableofcontents
\listoffigures
\listoftables
\lstlistoflistings
\newpage


\include{introduction}
\section{Requirements}

\subsection{Base Station Requirements}
All requirements are to be assumed to be of the base station (``the system'') unless explicitly stated otherwise.

\subsubsection{Functional Requirements}
\begin{outline}[enumerate]
\1 Shall be capable of upgrading its software and firmware upon administrator demand.
\1 Shall be capable of connecting with other PICA sub-systems.
\1 Shall receive and store power usage measurements from connected PICA systems.
\1 Shall function as a Network Time Protocol (NTP)server for connected PICA systems.
\1 Shall receive and record event log information from connected PICA systems.
\1 Shall be capable of connecting to a Local Area Network (LAN).
\1 Shall be configurable by the user.
\1 Shall be capable of displaying status information for connected PICA systems.
\1 Shall be capable of authenticating administrative access to connected PICA systems.
\1 Shall be capable of distributing system updates for connected PICA systems.
\1 Shall be capable of giving debugging and troubleshooting output.
\1 Shall be capable of actively notifying the power-company and consumer.
\end{outline}
\subsubsection{Behavioral Requirements}
\begin{outline}[enumerate]
\1 Shall store user-defined configuration in non-volatile media.
\1 Shall include a backup firmware in the event of a failed firmware upgrade.
\1 Shall store critical event logs from connected PICA systems in a non-volatile media.
\1 Shall host a webpage to display system information when browsed over LAN
\1 Shall store its software in non-volatile media.
\1 Shall run an operating system to manage hardware, device drivers, and connections to connected PICA systems.
\1 Shall use ZigBee to communicate with connected PICA systems.
\end{outline}

\subsubsection{Software Requirements}
\begin{outline}[enumerate]
\1 Shall include and run an upgradable operating system (OS).
\2 The OS shall include the drivers necessary to operate the system hardware.
\2 The OS shall include the protocols necessary to connect to PICA sub-systems.
\2 The OS shall be able to detect and identify connected PICA systems
\1 The OS shall maintain a list or database of PICA systems which the user has allowed.
\1 The OS shall control its connectivity hardware to prevent unwanted systems, such as those owned by other customers, from being connected to the system.
\1 The OS shall have an administrator-privileged user who may change the configuration of the system and of connected PICA systems.
\1 The OS shall give administrator privileges to the power-company and customer as specified by the system application.
\1 The OS shall include the protocols necessary to connect to the LAN.
\2 The OS shall include a DHCP client.
\2 The OS shall support both IPv4 and IPv6.
\1 Shall include and run a web server to provide the web interface.
\1 Shall include the necessary tools to download and apply software and firmware updates.
\end{outline}

\subsubsection{Hardware Requirements}
\begin{outline}[enumerate]
\1 Shall include an external power source.
\1 Shall be tolerant of fluctuations in input voltage.
\1 Shall have a central processor to execute software.
\1 Shall have adequate random-access memory (RAM) to execute software.
\1 Shall have an Ethernet controller for connecting over LAN.
\1 Shall have an RS-232 controller for debugging and troubleshooting the system.
\1 Shall have ZigBee connectivity hardware for communication with connected PICA systems.
\1 Shall have non-volatile storage dedicated to storing system firmware.
\1 Shall have non-volatile storage sufficient to store system software.
\1 Shall have non-volatile storage sufficient to store recorded events and short-term consumption history for up to a period of 3 years.
\end{outline}

\subsubsection{User Interface Requirements}
\begin{outline}[enumerate]
\1 Shall provide a web interface for viewing collected data over LAN.
\1 Shall provide a web interface for system administration to an authenticated administrator.
\2 Shall include an interface for managing connections to connected PICA systems.
\2 Shall include an interface for administration of connected PICA systems.
\3 Shall include interfaces for managing configurations of connected PICA systems.
\3 Shall include an interface for deploying software/firmware upgrades to connected PICA systems.
\1 Shall provide a debugging interface over an RS-232 serial connection.
\end{outline}

\subsubsection{Power Requirements}
\begin{outline}[enumerate]
\1 Shall be powered from a standard 120V wall outlet.
\1 Shall have a DC power supply to power internal components.
\1 Shall have a backup power supply.
\1 Shall require less than 10W to operate.
\end{outline}

\subsubsection{Codes and Compliances}
\begin{outline}[enumerate]
\1 Shall be UL certifiable, including all power supply hardware.
\1 Shall have a polarized electrical plug if the power supply is controlled with a switch.
\1 Shall restrict electromagnetic (EM) radiation to comply with FCC Title 47 Part 15.
\end{outline}

\subsection{Solid State Breakers and Monitoring Requirements}
All requirements are to be assumed to be of the solid state breakers (``the system'') unless explicitly stated otherwise.

\subsubsection{Functional Requirements}
\begin{outline}[enumerate]
\1 Shall be capable of completely interrupting power delivery on the connected  circuit. 
\1 Shall provide two-way communications to the Master Control Unit (MCU).
\1 Shall be capable of detecting brownout conditions.
\1 Shall temporarily store gathered information for transmission.
\1 Shall package the stored information for transmission over an ethernet link to MCU.
\1 Shall be capable of turning off circuits individually.
\1 Shall interrupt service to a circuit when current flow exceeds a specified threshold.
\end{outline}

\subsubsection{Behavioral Requirements}
\begin{outline}[enumerate]
\1 Shall be powered by line-voltage.
\1 Shall initialize all components when brought out of standby.
\1 Shall report all events to the critical event log.
\1 Shall monitor voltage levels in the connected circuit.
\1 Shall monitor current flow in the connected circuit.
\1 Shall monitor the number of kilowatt-hours used.
\1 Shall monitor the status of the power supply.
\end{outline}

\subsubsection{Hardware Requirements}
\begin{outline}[enumerate]
\1 Shall use non-volatile storage to store data when the system is without power. 
\1 Shall have a microcontroller for managing internal and external data and functions.
\1 Shall have a control panel external to the breaker box.
\1 Shall provide clocks for all synchronous components.
\end{outline}

\subsubsection{User Interface Requirements}
\begin{outline}[enumerate]
\1 Shall have a self-explanatory external interface.
\1 Shall have the ability to lockout the control panel. 
\end{outline}

\subsubsection{Power Requirements}
\begin{outline}[enumerate]
\1 Shall not restrict the flow of power to a circuit, except when a fault is detected.
\1 Shall be powered from before the breakers.
\end{outline}

\subsubsection{Mechanical Requirements}
\begin{outline}[enumerate]
\1 Shall fit into a standard, unmodified electric panel.
\end{outline}

\subsubsection{Safety Requirements}
\begin{outline}[enumerate]
\1 Shall provide circuit interrupter protection. 
\1 Shall have safety hazards clearly marked and visible from outside the system.
\1 Shall safely isolate high-voltage areas.
\end{outline}

\subsubsection{Codes and Compliances}
\begin{outline}[enumerate]
\1 Shall be compliant with ANSI C12.19.
\1 Shall be compliant with ANSI C12.21
\1 Shall be compliant with FCC Title 47 Part 15.
\end{outline}

\subsection{Electric Panel Meter Requirements}
All requirements are to be assumed to be of the electric panel meter (``the system'') unless explicitly stated otherwise.

\subsubsection{Functional Requirements}
\begin{outline}[enumerate]
\1 Shall continuously monitor the power used from either a single-phase or a multi-phase installation.
\1 Shall store power usage data locally, to be transmitted back to the base station at regular intervals.
\1 Shall display power usage data on an LCD display module integrated into the electric panel.
\1 Shall provide two-way communication with the Master Control Unit (MCU) to report usage data.
\1 Shall be capable of establishing a wireless link with a PICA base station.
\1 Shall be capable of detecting a brownout condition and storing critical data before shutting down.
\1 Shall be capable of restoring its last state after a brownout condition.
\1 Shall be capable of detecting any tampering and alerting both the power-company and the consumer.
\1 Shall monitor current flow through the main service lines.
\1 Shall monitor voltage levels on the main service lines.
\1 Shall control the service shutoff switch by receiving and validating a service shutoff message from the power-company.
\1 Shall provide a method for controlling the service shutoff switch from a local interface.
\1 Shall provide an interface for 3rd party meters, such as gas, water, or other utility meters to report data over the PICA network.
\1 Shall support on-demand reports of power usage, energy consumption, demand, power quality and system status.
\1 Shall support bi-directional metering and calculation of net power usage.
\1 Shall support automatic meter reads.
\1 Shall monitor the voltage flicker.
\1 Shall monitor the reactive power.
\end{outline}

\subsubsection{Behavioral Requirements}
\begin{outline}[enumerate]
\1 Shall, in the event of wireless link loss; attempt to re-establish the wireless link.
\1 Shall, in the event of a wireless link loss, revert to stand-alone mode, storing data internally until internal storage is full, at which point the system will begin overwriting the oldest data with the newest data.
\1 Shall, in the event of a wireless link loss, notify the user via the LCD display.
\1 Shall perform a built-in self-test upon system boot up.  
\1 Shall, in the event of a brownout, save all volatile information to non-volatile storage space.
\1 Shall be capable of detecting corrupted data when brought out of a brownout condition.
\1 Shall log all events processed in the following 4 categories: critical, error, warning, and note.
\1 Shall report all events to the PICA base station.
\1 Shall report to the power-company as specified by event criticality.
\1 Shall have dedicated non-volatile storage for all critical settings and configuration data.
\1 Shall compute the total power used in kilowatt-hours.
\1 Shall be capable of receiving messages from the power-company, providing the user with the current cost of a kilowatt-hour.
\1 Shall use 128-bit AES encryption for all messages transmitted outside of the device.
\1 Shall report the total amount of outage time.
\1 Shall date-stamp all detected outages with the date and time of the outage.
\end{outline}

\subsubsection{Software Requirements}
\begin{outline}[enumerate]
\1 Shall have enough onboard EEPROM for boot.
\1 Shall verify system firmware on boot up.
\1 Shall periodically perform a system check to verify the health and status of the system.
\1 Shall perform an on-demand system health and status check as demanded by the PICA base station or the power-company.
\1 Shall contain sufficient non-volatile storage for all system configuration settings.
\1 Shall be updateable through the power-company wireless interface.
\1 Shall be capable of properly recovering from a failed software update.
\1 Shall give authorized access to components of the system configuration as appropriate to the power-company and consumer.
\1 Shall notify the power-company and the PICA base station once service has been restored containing the time of restoration and a voltage measurement.
\1 Shall have a unique IPv6 address for the power-company mesh network.
\1 Shall have a unique IPv4 or IPv6 network address for the local home-area-network.
\1 Shall receive an NTP message from the PICA base station to set the hardware clock.
\1 Shall synchronize the hardware clock with the base station time once per day.
\1 Shall support on-demand hardware clock synchronization.
\end{outline}

\subsubsection{Hardware Requirements}
\begin{outline}[enumerate]
\1 Shall be completely enclosed in a weatherproof case, tolerant of extreme temperature differences.
\1 Shall be completely AC coupled against transient AC voltages.
\1 Shall be mounted in the same location as a standard power meter.
\1 Shall provide non-volatile storage.
\1 Shall be grounded.
\1 Shall provide a hardware system clock, set by the software and synchronized with the PICA base station.
\end{outline}

\subsubsection{User Interface Requirements}
\begin{outline}[enumerate]
\1 Shall have a 160-segment LCD display module, viewable from outside the electric panel.
\1 Shall be capable of interfacing with a web-based application for stand-alone configuration.
\1 Shall provide push-buttons for viewing contents of the display module.
\end{outline}

\subsubsection{Power Requirements}
\begin{outline}[enumerate]
\1 Shall be capable of operating from line-voltage.
\1 Shall be powered from before the master breaker, preventing the meter from losing power when the master breaker is switched off.
\end{outline}

\subsubsection{Safety Requirements}
\begin{outline}[enumerate]
\1 Shall meet or exceed safety requirements for devices inside an electric panel.
\1 Shall be AC coupled against incoming line-voltage and current.
\1 Shall be grounded.
\1 Shall be sealed against the elements.
\1 Shall safely isolate high-voltage areas.
\end{outline}

\subsubsection{Codes and Compliances}
\begin{outline}[enumerate]
\1 Shall be compliant with ANSI C12.19.
\1 Shall be compliant with ANSI C12.21.
\1 Shall be compliant with FCC Title 47 Part 15.
\end{outline}



\include{designGoals}
\include{systemDesign}
\include{majorDesignDecisions}
\include{hardwareDesign}
\include{softwareDesign}
\include{mechanicalDesign}
\section{Verification and Testing Plan}

\include{projectManagement}
\section{Business Plan}
\subsection{Introduction}
Many home and business owners have received electric bills and have sometimes wondered why it was so high. They generally wonder if they just simply used more energy that month or if one habit or appliance was to blame. However, they rarely see these questions answered. The PICA system will answer these questions without needing to go through the hassle of calling the power company and having a time consuming conversation. This feasibility study was conducted in order to gauge if the design team could develop this system in the time table given.  

\subsection{System Feasibility}
The PICA system separates into three subsystems that will communicate with each other to form one system. The team of four electrical engineers brings many different talents, experiences, and interests to contribute to the project. Each subsystem divides to cater to each team member’s area of interest and talent. By separating the system into different subsystems, it increases the feasibility of the system as a whole. This allows for each team member to do research, testing, and assembly separately. It also allows each team member to gauge time for parts to be delivery. The final step will be making sure each subsystem can communicate with each other. However, if the design team addresses these issues early, there should be fewer problems in the future. 

\subsection{Market Feasibility}
Not only will the PICA system monitor total power consumption, but it will be monitoring power on a circuit-by-circuit basis. This functionality distinguishes the PICA system from anything else on the market now, which only allow for plug-by-plug consumption or whole home consumption. Circuit-by-circuit monitoring allows the consumer to identify their energy hogs without having to test each device individually. This functionality also makes the market feasibility much greater. Another feature that increases the market feasibility is the fact that the PICA system will also have the capability of notifying the consumer when peak pricing hours of the day are, which is when the power company charges more per kilowatt-hour. 

\subsection{Legal Feasibility}
The PICA system will comply with all necessary codes and compliances, as well as keep the customer’s information private. Many different implementations accomplish these legal requirements. The operating system will only give administrator privileges to the power-company and customers, which the application of the system specifies. The user interface will have a web interface for system administration, but the information will only be accessible to an authenticated administrator. The PICA system also needs to meet certain codes in order to be safe enough for the customer to use, which will also protect from unexpected lawsuits. Underwriters Laboratories (UL) is an independent product safety certification organization, which offers safety certifications to products\cite{UL_Web}.  The system will be UL certifiable. The system will also restrict electromagnetic (EM) radiation to comply with FCC Title 47 Part 15. It will also be compliant with ANSI C12.19 and ANSI C12.21 standards. The PICA system will also safely isolate high-voltage areas.
 
\subsection{Schedule Feasibility}
The timeline of the project is eight months. During these eight months, the team members will develop a prototype to test and make necessary corrections based on testing results. The feasibility of this schedule is within the range of time needed to complete this project. Team members holding other team members accountable for their respective tasks maintain this schedule. The system divides into sub-systems for each team member to work on also contributes to a feasible eight-month schedule. The team has milestones in a Gantt chart to help keep an accurate account of the timeline. Additionally, there is a project manager specifically responsible to keep tasks updated and keep team members on track.

\subsection{Resource Feasibility}
There are many resources available to the team that can contribute to the project. There are many different facilities to have space to work. People and company resources are great for the design team as well, because the experience level is much higher. Consumer's Energy has offered to be available for questions, should any arise. Mark Mitchmerhuizen is head engineer at Johnson Controls Incorporated and has volunteered to be a mentor to the design team. Calvin College has an electronic shop with free parts that are available for students to use. For the parts that are not available in the electronics shop, Calvin College gives the design team a three hundred dollar budget to use at their disposal. This could be a potential problem to feasibility, because due to the size of the project, the cost might exceed three hundred dollars. To account for this the team members are acquiring some part donations from companies willing to help.


\section{Parts and Project Costs}

\include{acknowledgements}
\include{conclusions}
\bibliographystyle{IEEEtran}
\bibliography{IEEEabrv,bibliography}


\include{appendicies}
%\newpage
%\bibliographystyle{IEEEtran}
%\bibliography{IEEEabrv,ARM_Microprocessor}

\end{document}

%%%%%%%%%%%%%%%%%%%%%%%%%%%%%%%%%%%%%%%%%%%%%%%%%%%%%%%%%%%%%
